\PassOptionsToPackage{dvipsnames}{xcolor}

\documentclass[10pt,a4paper,ragged2e,withhyper]{altacv}

\geometry{left=1.25cm,right=1.25cm,top=1.5cm,bottom=1.5cm,columnsep=1.2cm}

\usepackage{paracol}
\usepackage{csquotes}
\usepackage{amssymb}
\usepackage{fontawesome}

\ifxetexorluatex
  \setmainfont{Roboto Slab}
  \setsansfont{Lato}
  \renewcommand{\familydefault}{\sfdefault}
\else
  \usepackage[rm]{roboto}
  \usepackage[defaultsans]{lato}
  \renewcommand{\familydefault}{\sfdefault}
\fi

\definecolor{SlateGrey}{HTML}{2E2E2E}
\definecolor{LightGrey}{HTML}{666666}
\definecolor{DarkPastelRed}{HTML}{450808}
\definecolor{PastelRed}{HTML}{8F0D0D}
\definecolor{GoldenEarth}{HTML}{E7D192}
\colorlet{name}{black}
\colorlet{tagline}{PastelRed}
\colorlet{heading}{DarkPastelRed}
\colorlet{headingrule}{GoldenEarth}
\colorlet{subheading}{PastelRed}
\colorlet{accent}{PastelRed}
\colorlet{emphasis}{SlateGrey}
\colorlet{body}{LightGrey}

\renewcommand{\namefont}{\Huge\rmfamily\bfseries}
\renewcommand{\personalinfofont}{\footnotesize}
\renewcommand{\cvsectionfont}{\LARGE\rmfamily\bfseries}
\renewcommand{\cvsubsectionfont}{\large\bfseries}

\renewcommand{\itemmarker}{{\small\textbullet}}
\renewcommand{\ratingmarker}{\faCircle}

\begin{document}
\name{Aleksei Shabanov}
\tagline{Machine learning engineer}

\personalinfo{
  \email{alexey.shabanoff@gmail.com}
  \phone{+7 (981) 711-82-02}
  \github{alekseysh}
  \location{Moscow, Russia}
}

\makecvheader

%% Set the left/right column width ratio
\columnratio{0.6}

\smallskip
I am a machine learning engineer who strives to find balance between complex research and quality code.

\begin{paracol}{2}

\cvsection{Work experience}

\cvevent{Machine learning engineer}{Neuromation Inc.}{March 2018 -- October 2020}{Saint-Petersburg, Russia}
Working in a start-up allowed me to broaden my horizons and realise projects in machine learning consulting, for example:
\begin{itemize}
\item Built a \textbf{computer vision} system which was able to save doctors' time by checking if a box with surgical tools was correctly completed. This was a challenge because there were a lot of diverse tools and we had to solve the problem without retraining the model (in a few-shot way), \href{https://drive.google.com/file/d/1FBNRTkxkGzzfVHlrwJCauK7xjBWkI7xO/view?usp=sharing}{\underline{link}}.
\item Created a \textbf{text} processing tool to extract useful information from emails. It required to fulfil a project from communicating with the client and setting up the data labelling process to containerization and deploying the trained model.
\end{itemize}

\divider

\cvevent{Junior data analyst}{3Red Trading, LLC}{June 2016 -- March 2017}{Saint-Petersburg, Russia}
The company analysed billions of transactions from the \href{https://en.wikipedia.org/wiki/Chicago_Mercantile_Exchange}
{Chicago Mercantile Exchange}.
Based on statistical anomalies, I recognized traders who used automatic tools, which allowed our team to predict their behaviour.
I also learned how to make reports with a large number of graphs as clear as possible.

\cvsection{Side projects}

\cvevent
{\href{https://github.com/catalyst-team/catalyst}{Catalyst}}
{In top-10 contributors by code amount}{}{}
Catalyst is a popular \textbf{open-source} analogue of Keras for PyTorch (2.3k stars on GitHub, part of PyTorch ecosystem).
Among other things, I have implemented and tested a metric learning module for obtaining the representations of objects.
\smallskip

\href{https://medium.com/pytorch/metric-learning-with-catalyst-8c8337dfab1a}{\underline{Medium post} [en]}. \href{https://www.youtube.com/watch?v=a7sDJMDatZ4}{\underline{Video from Data Fest} [en]}.

\divider

\cvevent{\href{https://zindi.africa/competitions/uber-movement-sanral-cape-town-challenge}{Uber Movement Challenge}}
{2nd place (out of 100+ teams)}
{11 Oct 2019 -- 10 Feb 2020}{}
This was a tabular competition with the following task: \textit{"Predict when and where road incidents will occur next in Cape Town"}. We achieved good results due to non-standard feature engineering for spatial data and a neat validation pipeline.
\smallskip

\href{https://github.com/AlekseySh/uber_competition}{\underline{GitHub}}.
\href{https://youtu.be/CliQ-EiYv5A}{\underline{Video from Data Fest} [ru]}.

\divider

\cvevent
{\href{https://research.jetbrains.org/groups/plt_lab/seminars}{JetBrains seminars at Computer Science Center}}
{Gave a lecture about Person Re-Id task}
{}
{}
{\href{https://youtu.be/O8qtBYeOSKE}{\underline{Video} [ru]}.}

\switchcolumn

\cvsection{Skills and abilities}
\cvtag{nlp}
\cvtag{cv}
\cvtag{tabular analysis}
\cvtag{Re-Id}

\cvtag{segmentation}
\cvtag{tracking}
\cvtag{metric learning}
\divider

\cvtag{Python}
\cvtag{PyTorch}
\cvtag{OpenCV}
\cvtag{NumPy}
\cvtag{pandas}
\cvtag{sklearn}
\cvtag{CatBoost}
\cvtag{Jupyter}

\divider

\cvtag{Linux}
\cvtag{git}
\cvtag{CI/CD}
\cvtag{docker}

\cvtag{SQL (basic)}
\cvtag{MATLAB}
\cvtag{\LaTeX}

\cvsection{Education}

\cvevent{MSc in Applied Mathematics and Physics}{Saint-Petersburg State University}{2016 -- 2019}{}
Thesis title: "Semi-automatic cataloguing of images in a social network using machine learning methods".

\divider

\cvevent{BSc in Physics}{Saint-Petersburg State University}{2012 -- 2016}{}
Thesis title "Statistical methods for the separating of interfering seismic waves".

\cvsection{Courses}
\begin{itemize}

\item Introduction to databases, \href{https://stepik.org/cert/336064}
{\underline{certificate}}.

\item Functional programming in Haskell,
\href{https://stepik.org/cert/272055}
{\underline{certificate}}.

\item Graph theory,
\href{https://coursera.org/share/713ce9741f5f39a7c4d56af9d5e9b60e}
{\underline{certificate}}.

\end{itemize}

\cvsection{Languages}
\begin{itemize}
\item English — Upper-Intermediate (B2)
\item Russian — Native speaker
\end{itemize}


\end{paracol}

\end{document}
